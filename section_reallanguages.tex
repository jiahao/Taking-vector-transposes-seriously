\label{sec:reallanguages}
\section{Real programming languages}

\subsection{The polymorphic algebra of {*}: composition and broadcasting}

Another legacy of the post ALGOL 60 working committee was the {*}
symbol.\cite{Hockney1961}

As we all know, matrix multiplication and its derivative products
(matvec, outer, etc) has its origins in the composition of linear
maps, which historically arose from considering substitutions of variables
in systems of linear equations. At the same time, there is an entirely
different kind of product, that of broadcasting, which is scalar-vector
product and scalar-matrix product. Scalar multiplication is an axiom
associated with vector space structure, whereas the composition of
two matrices is very much the province of group theory and abstract
algebra. Interestingly the latter is usually about square matrices
though...

Scalar multiplication semantics has evolved today to much more complicated
operations, such as broadcasting operations, which are very much ``tensor
flavored'' and are incompatible with the ordinary matrix product.
(The equivalent operation between two matrices goes by the name Hadamard
product.) MATLAB uses .{*}, Eigen uses cwiseProduct.

We won't focus very much on general broadcasting products other than
to acknowledge that it is very much ``arraylike semantics'' and
complicates the discussion beyond one and two dimensional objects.
For our purposes it will suffice to note that almost all modern languages
have an operator or function that is ALGOL's {*} that has both compositional
and broadcasting semantics.\footnote{A notable exception is APL, which clearly distinguished broadcasting
products in multiply $\times$ (via what is called ``scalar extension''
in APL, i.e. broadcasting) from the compositional product $+.\times$.}

To tease apart the semantics more clearly, we will denote compositional
products like matrix multiply by $*$ and broadcasting products by
$\circledast$. The circle around the latter symbol will remind us
that it has pointwise semantics, just like the . in MATLAB's .{*}.
Where convenient we will also subscript each product operation by
the type of its operands, be they \uline{s}calar, \uline{v}ector
or \uline{m}atrix.


\subsection{A note on the tensor transpose}

Does the general tensor transpose allow us to reconcile these semantics?
Yes and no.

The NumPy definition $\cdot_{v}^{T}$ is compatible with the tensor
transpose, which really a permutation of indices. APL also uses this
``reverse dims'' definition of transpose in its ⍉ operator. But
note that the tensor object here is really constructed only from ``column
vectors'',
\[
F^{m_{1}\times m_{2}\times\dots\times m_{r}}=F^{m_{1}}\times F^{m_{2}}\times\dots\times F^{m_{r}}.
\]
You can express a 2D array as $F^{m}\times F^{n}$, but now you index
a matrix with $\sum_{kl}A_{kl}\left(e_{i}\right)_{k}\left(e_{i}\right)_{l}=A_{ij}$
which really can't be expressed in linear algebra form like $e_{i}^{\prime}Ae_{j}$.
That matrix instead is $F^{m}\times\tilde{F}^{n}$, which requires
a more general definition of tensor that allows an arbitrary $\times$-combination
of $F$s and $\tilde{F}$s. But then an implementation of this data
type needs to store not only the rank $r$ but also the ``up- or
downness'' representing the exact sequence of $F$ and $\tilde{F}$s.
Furthermore, you now see that the tensor transpose $\cdot_{t}^{T}$
now means something like
\[
\cdot_{t}^{T}:F^{m}\times\tilde{F}^{n}\rightarrow F^{n}\times\tilde{F}^{m}
\]

\[
\cdot_{t}^{T}:F^{m}\rightarrow F^{m}
\]
associated with the permutations $\overline{21}$ and $1$ respectively,
with the overline $\overline{\cdot}$ over each index meaning that
$F$ goes to $\tilde{F}$ and vice versa. It is no longer the simple
permutation 21. Furthermore you now get other ``transposes'' associated
with the permutations $\overline{2}1$ and $2\overline{1}$, neither
of which are matrices. We speculate that forcing users to worry about
up-or-downness is undesirable.


\subsection{There be problems}


Problems arise when trying to implement linear algebra on arrays, which are
data structures defined by their indexing behavior. In particular, array data
structures have a specific dimensionality specifying the number of indexes
needed to address elements in the array~\footnote{The array is such a
fundamental data structure that it is difficult to pin down a concrete
definition. Nevertheless, the early implementations of arrays in languages such
as FORTRAN~\cite[p36]{Backus1956} and ALGOL~60~\cite[Sec. 2.3]{Randell1964}
make clear that indexing by a fixed number of integers is the most primitive
operation associated with an array.}. While in the mathematical literature the
distinction between ``column vector'' and ``column matrix'' is often glossed
over, these terms now refer to two different possible representations as one-
and two-dimensional arrays respectively. If a column vector can be represented
as a one-dimensional array, there is also the question of whether ``row
vectors'' can be meaningfully represented using one-dimensional arrays also.
Furthermore, there is the question of whether a scalar product, by analogy with
matrix multiplication, should return a true scalar (a 0-dimensional array) or a
$1\times1$ matrix (a two-dimensional array).

It turns out that various programming languages and libraries used for
numerical computing implement many different approaches to the problem of
representing vectors.

In this section, we cover what programming languages do.

\subsection{Methodology}

We considered a variety of real world programming languages (with suitable
libraries).  For each system, we considered the following questions:

\begin{enumerate}

\item
Does this programming language allow users to provide distinct types for
scalar, vector and matrix variables?

\item
Are the multiplication, division and transposition operators defined for all
possible types of variables?

\item
Can we use these multiplication, division and transposition operators to
construct expressions that conform to the standard way of spelling them in
numerical linear algebra and have the expected output? (For example, does
\verb`x'*y` compute the inner product of \verb`x` and \verb`y`, producing a
scalar?)

\item
Are standard identities in numerical linear algebra expressible in code form?

\end{enumerate}

\begin{tabular}{|l|l|l|l|l|l|l|l|l|l|l|}
\cline{2-11}
\multicolumn{1}{l|}{} & \multicolumn{10}{l|}{ELEMENTARY BEHAVIORS}\tabularnewline
\cline{2-11}
\multicolumn{1}{l|}{} & A{*}B & A{*}b & 1{*}a & a{*}1 & a{*}b & a{*}B & A' & a' & A/B & 1/1\tabularnewline
\midrule
Julia & M & V & V & V & E & M{*} & M & M & M & S\tabularnewline
\midrule
Lua/Torch & M & V & \textbf{E} & V & S & E & M & E & E & S\tabularnewline
\midrule
Mathematica & M & V & V & V & E & V & M & V & M & S\tabularnewline
\midrule
Matlab & M & - & - & - & - & - & M & - & M & -\tabularnewline
\midrule
NumPy & M & V & V & V & S & M & M & V & E & S\tabularnewline
\midrule
Python 3.5 & M & V & V & V & S & M & M & V & E & S\tabularnewline
\midrule
R & M & M & - & - & E & M{*} & M & M & M & -\tabularnewline
\midrule
\end{tabular}

M: Matrix, V: Vector, S: Scalar, E: Error, M{*}: matrix only if second
argument is 1xN, otherwise error, -: Concept does not exist

\textbf{Bold} : Most people consider this wrong behavior

\begin{tabular}{|l|l|l|l|l|l|l|l|}
\cline{2-8}
\multicolumn{1}{l|}{} & \multicolumn{4}{l|}{EXPRESSIONS} & \multicolumn{3}{l|}{IDENTITIES}\tabularnewline
\cline{2-8}
\multicolumn{1}{l|}{} & a'{*}b & a{*}b' & x'{*}A{*}y & x'{*}A{*}x/x'{*}x & (x{*}y'){*}z = x{*}(y'{*}z) & x''=x & (A{*}x)'=(x'{*}A')\tabularnewline
\midrule
Julia & \textbf{V} & M & \textbf{V} & \textbf{E} & \textbf{E} & F & T\tabularnewline
\midrule
Lua/Torch & S & \textbf{S} & E & \textbf{E} & \textbf{E} & T & \textbf{F}\tabularnewline
\midrule
Mathematica & S & \textbf{S} & S & S & V & T & T\tabularnewline
\midrule
Matlab & M & M & M & M & M & T & T\tabularnewline
\midrule
NumPy & S & \textbf{S} & S & S & V & T & T\tabularnewline
\midrule
Python 3.5 & S & \textbf{S} & S & S & V & T & T\tabularnewline
\midrule
R & M & M & M & M & M & \textbf{F} & T\tabularnewline
\midrule
\end{tabular}

T: True, F: False



\subsection{C++/Eigen}

Eigen is a C++ library for high performance numerical linear
algebra.\cite{eigenweb}\footnote{The version of Eigen tested was v3.2.6,
compiled with clang with Apple LLVM v7.0.2. Eigen supports vectors and
matrices of fixed sizes (known at compile time) and dynamic sizes (known only
at run time). Here, we consider only the dynamic classes, which have
the most intricate semantics.
}

Eigen heavily exploits C++ expression templates to provide flexibility
resulting from type polymorphism. Notably, it uses special templates (meant for
internal use) to capture the result of intermediate expressions, performing
what computer scientists call ``lazy evaluation''. For example, the result of
the expression \verb`x.transpose()`, performing the transpose of a vector
\verb`x` (of type \verb`VectorXd`) is neither a vector \verb`VectorXd` nor a
matrix \verb`MatrixXd`, but rather a special type, \verb`Transpose<`
\verb`VectorXd>`, that can be assigned to a new \verb`VectorXd` variable or a
new \verb`MatrixXd` variable. Similar types are used to capture and build other
intermediate expressions, such as \verb`ProductReturnType<MatrixXd, VectorXd>`
for the matrix-vector product \verb`A*x`.  This ingenious strategy of
determining the output type only at assignment time is made possible in a
static language like C++, where users are expected to specify the output type
of a variable when declared.\footnote{Strictly speaking, C++ also allows for
run-time polymorphism, where the output type is only determined at run time.
However, the mechanisms for doing so are cumbersome and in most cases produce
significant performance slowdowns.} The result is that in almost all cases,
users can choose to imagine the result of most linear algebra expressions as
vectors or matrices as they so choose.

Nevertheless, Eigen's strategy is not perfect. The heavy use of internal
classes to represent the output of computations means that both assignment and
equality testing in Eigen must be overloaded with new definitions to meet user
expectations. Oftentimes, the validity of an assignment statement can only be
determined at run time, resulting in overhead due to run time checks for
compatible shape and so on. Many new methods have to be introduced to overload
C++'s assignment and equality testing operators. Furthermore, the assignment
logic of vector transpose is ambiguous and may lead to subtle logic errors
because the result of assignment may violate user expectations of the
transitivity of equality. In Eigen, the result of \verb`x.transpose()` can be
assigned to either a \verb`VectorXd` variable or \verb`MatrixXd` variable,
i.e., both of the following statements are valid:

\begin{verbatim}

VectorXd y = x.transpose();
MatrixXd Y = x.transpose();

\end{verbatim}
%
However, the shapes of \verb`y` and \verb`Y` differ: the former has $n$ rows
and 1 column\footnote{Eigen defaults to column vectors and employs an implicit
trailing singleton dimension rule similar to \sc{MATLAB}'s.} whereas the latter
has 1 row and $n$ columns. In other words, the type of the variable being
assigned to determines whether transposing a vector is the identity operation
or produces a row matrix!

We find that unexpected behavior may result from the logic above. For example,
the identity ${(Ax)}^\prime = x^\prime A^\prime$ may or may not hold depending on
whether the expressions on either side have been assigned to variables, and if
so, which variables:
%
\begin{verbatim}

//The type of (A*x).transpose() is
//  Transpose<GeneralProduct<MatrixXd, VectorXd>>
VectorXd v1 = (A*x).transpose();
MatrixXd m1 = (A*x).transpose();

//The type of x.transpose() * A.transpose() is
//  ProductReturnType<Transpose<VectorXd>, Transpose<MatrixXd>>
VectorXd v2 = x.transpose() * A.transpose();
MatrixXd m2 = x.transpose() * A.transpose();

assert ((A*x).transpose() == x.transpose() * A.transpose()); //true
assert (v1 == v2); // true
assert (m1 == m2); // true
assert (v1 == m2); // assertion failure
assert (m1 == v2); // assertion failure

\end{verbatim}
%
The equality comparison operator \verb`==` must be suitably defined to
understand when two values with complicated and different types can test equal
to each other in the three cases above that do not trigger the assertion
failure. Furthermore, the result of assigning \verb`(A*x).transpose()` to
\verb`v1` produces a column vector effectively equal to \verb`A*x`, but the
assignment to \verb`m1` produces a row matrix that is the transpose of
\verb`v1`.
A similar result arises for \verb`v2` and \verb`m2`, with the result that the
equality holds only when neither side is assigned to a vector or matrix
variable, or when both sides are assigned to variables of the same type. In
this case, users may be confused as to why certain quantities may not be equal
to each other as a result of subtle reasoning errors about the types of various
expressions.


\subsection{Julia}

Julia is a dynamic for technical computing, whose base library features
generous functionality for linear algebraic operations.~\cite{Bezanson2012}

Julia's multiplication operator, \verb`*`, implements matrix--matrix and
matrix--vector products, as well as scalings of vectors and matrices, on top of
ordinary scalar multiplication.
Julia's division \verb`x/y` is equivalent to multiplication of \verb`x` by the
inverse of \verb`y` on the right. For non-scalar \verb`y`, the inverse is
generalized to matrix inverses and pseudoinverses (in the least-squares sense)
where appropriate.
Julia's transposition operator, \verb`'`, promotes vectors to column matrices
before transposing them. Thus we have that vector transposition is not
idempotent, \verb`x''`$\ne$\verb`x`.

Julia's current semantics lead to some unexpected behavior. Expressions like
\verb`a'*b` for the inner product produce a vector of length 1, which do not
have the semantics of scalars. Other identities, like $(xy^\prime)z =
x(y^\prime z)$, also fail to hold in code, since \verb`x*(y'z)` becomes the
product of a vector with a 1-vector, which is not defined.


\subsection{Lua/Torch}

Torch is a Lua\cite{Lua} package for machine learning applications.~\cite{Torch}
Although not designed for linear algebra per se, Torch uses numerical linear
algebra routines heavily under the hood for fast computations.

The key data structure of Torch is its \verb`Tensor` class, which implements
the semantics of multidimensional arrays. Unlike the other languages considered
here, Torch only defines transposition on matrices (tensors with two
dimensions). As a result, most of the analysis about expressions and identities
using vector transposes cannot be applied, as any expression involving a vector
transpose produces an error.
Instead, Torch requires the \verb`dot` method to write expressions such as the inner product,
\verb`x:dot(y)`. \verb`x:dot(y)` can be thought of as the composition of
\begin{enumerate}

\item
Reshaping the vector \verb`x` to a row matrix \verb`X`,

\item
Computing the matrix--vector product \verb`z = X*y`,

\item
Reshaping the 1-vector \verb`z` to a scalar.

\end{enumerate}

Another quirk of Torch is that binary expressions of a scalar and a tensor must
have the tensor operand come first, due to limitations of the underlying Lua
language, which does not allow packages to define new methods on built-in
types such as floating-point numbers. Thus care must be taken to write
scalar--matrix products in matrix--scalar order.


\subsection{R}

R is a widely used language for statistical computing.~\cite{Rlang}.
R's statistical features work primarily on data frames, time series and vectors.
R also has matrices, but no true scalars.
Its matrix multiplication operator is \verb`%*%` and its transposition operator
is \verb`t()`.
R's semantics promotes $n$-vectors automatically to matrices whenever linear algebraic
operations like \verb`t()` or \verb`%*%` are encountered.
By default, $n$-vectors behave as if they were $1\times n$ row matrices, except in
transposition, which produces a row matrix, and in matrix--vector products, which
produce a column matrix.
Like Python, \verb`%*%` is also defined on two vectors to be the dot product.
However, unlike Python, \verb`%*%` has a special case where if the second operand
is a length 1 vector, the result is a $1\times n$ matrix.
Extraneous singleton dimensions can be removed with \verb`drop()`, which converts
either $n\times1$ or $1\times n$ matrices into $n$-vectors.

R's semantics imply that transposition is not idempotent on vectors.
Nevertheless, we've already seen precedent in the mathematical literature
that does...
\todo{cite}

\section{Embedding calculus}

It turns out that various programming languages and libraries used
for numerical computing implement many different approaches to the
problem of representing vectors.

We considered a variety of real world programming languages (with
suitable libraries). For each system, we considered the following
questions:
\begin{enumerate}
\item Does this programming language allow users to provide distinct types
for scalar, vector and matrix variables?
\item Are the multiplication, division and transposition operators defined
for all possible types of variables?
\item Can we use these multiplication, division and transposition operators
to construct expressions that conform to the standard way of spelling
them in numerical linear algebra and have the expected output? (For
example, does \verb`x'*y` compute the inner product of \verb`x`
and \verb`y`, producing a scalar?)
\item Are standard identities in numerical linear algebra expressible in
code form?
\end{enumerate}
\begin{table*}
\caption{Summary of semantics across programming languages}
\begin{tabular}{lllllllllll}
\cmidrule{2-11}
\multicolumn{1}{l}{} & \multicolumn{10}{l}{ELEMENTARY BEHAVIORS}\tabularnewline
\cmidrule{2-11}
\multicolumn{1}{l}{} & A{*}B  & A{*}b  & 1{*}a  & a{*}1  & a{*}b  & a{*}B  & A'  & a'  & A/B  & 1/1\tabularnewline
\midrule
Julia  & M  & V  & V  & V  & E  & M{*}  & M  & M  & M  & S\tabularnewline
\midrule
Lua/Torch  & M  & V  & \textbf{E}  & V  & S  & E  & M  & E  & E  & S\tabularnewline
\midrule
Mathematica  & M  & V  & V  & V  & E  & V  & M  & V  & M  & S\tabularnewline
\midrule
Matlab  & M  & -  & -  & -  & -  & -  & M  & -  & M  & -\tabularnewline
\midrule
NumPy  & M  & V  & V  & V  & S  & M  & M  & V  & E  & S\tabularnewline
\midrule
Python 3.5  & M  & V  & V  & V  & S  & M  & M  & V  & E  & S\tabularnewline
\midrule
R  & M  & M  & -  & -  & E  & M{*}  & M  & M  & M  & -\tabularnewline
\bottomrule
\end{tabular}

M: Matrix, V: Vector, S: Scalar, E: Error, M{*}: matrix only if second
argument is 1xN, otherwise error, -: Concept does not exist

\textbf{Bold} : Most people consider this wrong behavior

\begin{tabular}{llllllll}
\cmidrule{2-8}
\multicolumn{1}{l}{} & \multicolumn{4}{l}{EXPRESSIONS} & \multicolumn{3}{l}{IDENTITIES}\tabularnewline
\cmidrule{2-8}
\multicolumn{1}{l}{} & a'{*}b  & a{*}b'  & x'{*}A{*}y  & x'{*}A{*}x/x'{*}x  & (x{*}y'){*}z = x{*}(y'{*}z)  & x''=x  & (A{*}x)'=(x'{*}A')\tabularnewline
\midrule
Julia  & \textbf{V}  & M  & \textbf{V}  & \textbf{E}  & \textbf{E}  & F  & T\tabularnewline
\midrule
Lua/Torch  & S  & \textbf{S}  & E  & \textbf{E}  & \textbf{E}  & T  & \textbf{F}\tabularnewline
\midrule
Mathematica  & S  & \textbf{S}  & S  & S  & V  & T  & T\tabularnewline
\midrule
Matlab  & M  & M  & M  & M  & M  & T  & T\tabularnewline
\midrule
NumPy  & S  & \textbf{S}  & S  & S  & V  & T  & T\tabularnewline
\midrule
Python 3.5  & S  & \textbf{S}  & S  & S  & V  & T  & T\tabularnewline
\midrule
R  & M  & M  & M  & M  & M  & \textbf{F}  & T\tabularnewline
\bottomrule
\end{tabular}

T: True, F: False
\end{table*}


\subsection{Matlab}

Matlab is...

Matlab supports real and complex numbers, $F=\mathbb{R}$ or $\mathbb{C}$.
Reprsented by single or double (default) precision floating point
numbers.

Matlab has no true scalars or vectors. Everything is a matrix.

Matlab's definition of vectors, unlike most of the other languages
considered here, fall back to the older ``matrix first'' view of
linear algebra.
\begin{quotation}
``Matrices with one dimension equal to one and the other greater
than one are called vectors.'' - \cite{matlab-empty-mat}
\end{quotation}
The semantics of Matlab's {*} and ' operations are:
\begin{itemize}
\item {*}: $F^{m\times k}\times F^{k\times n}\rightarrow F^{m\times n}$
is matrix multiply, $*_{mm}$.
\item {*}: $F^{1\times1}\times F^{m\times n}\rightarrow F^{m\times n}$
is the scalar–matrix product, $\circledast_{sm}\circ\left(\theta\times I\right)$.
\item {*}: $F^{m\times n}\times F^{1\times1}\rightarrow F^{m\times n}$
is the matrix–scalar product, $\circledast_{ms}\circ\left(I\times\theta\right)$.
\item .': $F^{m\times n}\rightarrow F^{n\times m}$ is matrix transpose
$\cdot^{T}$
\item ': $F^{m\times n}\rightarrow\left(F^{*}\right)^{n\times m}$ is matrix
transpose with complex conjugation $\cdot^{\prime}=\cdot^{*}\circ\cdot^{T}$.
\end{itemize}
Results:
\begin{itemize}
\item Transpose is idempotent, since A'' = $\left(\cdot^{T}\circ\cdot^{T}\right)(A)=A$.
\item u'{*}v=$*_{mm}\circ\left(\cdot^{\prime}\times I\right)$ returns the
inner product as a $1\times1$ matrix.
\item u{*}v'=$*_{mm}\circ\left(I\times\cdot^{\prime}\right)$ returns the
outer product.
\item u'{*}A{*}v=(u'{*}A){*}v' because MATLAB evaluates from left to right.
So (u'{*}A){*}v'$=*_{mm}\circ\left(*_{mm}\times I\right)\circ\left(\cdot^{\prime}\times I\times I\right)$
computes the bilinear form as a $1\times1$ matrix.
\end{itemize}
Indexing behaviors:

Summary:

MATLAB has nearly pure matrix semantics $\theta$, $*_{mm}$, $\cdot^{T}$,
$I$. Since MATLAB only has matrices and no separate vector or scalar
types, the matrix-to-scalar embedding $\theta$ is used to convert
$1\times1$ matrices into scalars, and therefore even basic operations
like {*} in MATLAB must check the shape of the inputs before deciding
whether to do $*_{mm}$, $\circledast_{sm}$ or $\circledast_{ms}$.

\subsection{Python/NumPy}

NumPy's dot function has a nice rank rule, which is that
\begin{quotation}
Dot product of two arrays.

For 2-D arrays it is equivalent to matrix multiplication, and for
1-D arrays to inner product of vectors (without complex conjugation).
For N dimensions it is a sum product over the last axis of a and the
second-to-last of b. \cite{numpy.dot}
\end{quotation}
But even so, dot allows scalars! so this description is incomplete.
It only describes the $*$ semantics but not the $\circledast$ semantics!

NumPy also now has a matmul function which allows working on fields
of $F^{m_{1}\times...\times m_{r}}$. It implements the semantics
of the @ operator implemented in Python 3.5 following PEP 465 \cite{Smith2014}.
The definition of @ is that
\begin{quote}
1d vector inputs are promoted to 2d by prepending or appending a '1'
to the shape, the operation is performed, and then the added dimension
is removed from the output. The 1 is always added on the \textquotedbl{}outside\textquotedbl{}
of the shape: prepended for left arguments, and appended for right
arguments.
\end{quote}
The semantics described in PEP 465 can be transcribed directly into
our notation as follows:
\begin{itemize}
\item @: $F^{m\times k}\times F^{k\times n}\rightarrow F^{m\times n}$ is
matrix multiply, $*_{mm}$.
\item @: $F^{m\times n}\times F^{n}\rightarrow F^{n}$ is the matrix–vector
product, $*_{mv}$, decomposed explicitly as $*_{mv}=\phi^{-1}\circ*_{mm}\circ\left(I\times\phi\right)$.
\item @: $F^{m}\times F^{m\times n}\rightarrow F^{n}$ is the matrix–vector
product, $*_{vm}$, decomposed explicitly as $*_{vm}=\tilde{\phi}^{-1}\circ*_{mm}\circ\left(\tilde{\phi}\times I\right)$
where $\tilde{\phi}:F^{n}\rightarrow F^{1\times n}$ is the embedding
of an $n$-vector into a row matrix and is formally equivalent to
$\tilde{\phi}=\cdot^{T}\circ\phi$.
\item @: $F^{m}\times F^{m}\rightarrow F$ is the vector–vector product,
$*_{vv}$, decomposed explicitly as $*_{vv}=\theta\circ*_{mm}\circ\left(\tilde{\phi}\times\phi\right)$.
\item numpy.transpose: $F^{m}\rightarrow F^{m}$ is the identity, $I_{v}$.
\item numpy.transpose: $F^{m\times n}\rightarrow F^{n\times m}$ is the
matrix transpose, $\cdot^{T}$.
\end{itemize}
numpy.transpose implements the tensor transpose without up-or-down
semantics, defaulting to the permutation that reverses all the dimensions.

numpy.dot includes the same semantics of @ (which is implemented as
numpy.matmul), but additionally allows either argument to be a scalar,
hence including $\circledast$ semantics also. There are also differences
in handling arrays of higher rank which we shall not consider.

Consequence:
\begin{itemize}
\item u'@v is the inner product, $*_{vv}\circ\left(I_{v}\times I_{v}\right)=*_{vv}$.
\item u@v' is not the outer product, but is instead the inner product, $*_{vv}\circ\left(I_{v}\times I_{v}\right)=*_{vv}$.
\item u'@A@v is the bilinear form. Since Python evaluates operators from
left to right \footnote{with the exception of the exponentiation operator {*}{*} which we
do not consider here.}, u'@A@v=(u'@A)@v is the operation $*_{vv}\circ\left(*_{vm}\times I_{v}\right)\circ\left(I_{v}\times I\times I_{v}\right)$.
\end{itemize}
Python's @ semantics have the advantage of clearly separating compositional
product $*$ from broadcasting product $\circledast$ .

\subsection{C++/Eigen}

Eigen is a C++ library for high performance numerical linear algebra.\cite{eigenweb}\footnote{The version of Eigen tested was v3.2.6, compiled with clang with Apple
LLVM v7.0.2. Eigen supports vectors and matrices of fixed sizes (known
at compile time) and dynamic sizes (known only at run time). Here,
we consider only the dynamic classes, which have the most intricate
semantics. }

Eigen heavily exploits C++ expression templates to provide flexibility
resulting from type polymorphism. Notably, it uses special templates
(meant for internal use) to capture the result of intermediate expressions,
performing what computer scientists call ``lazy evaluation''. For
example, the result of the expression \verb`x.transpose()`, performing
the transpose of a vector \verb`x` (of type \verb`VectorXd`) is
neither a vector \verb`VectorXd` nor a matrix \verb`MatrixXd`, but
rather a special type, \verb`Transpose<` \verb`VectorXd>`, that
can be assigned to a new \verb`VectorXd` variable or a new \verb`MatrixXd`
variable. Similar types are used to capture and build other intermediate
expressions, such as \verb`ProductReturnType<MatrixXd, VectorXd>`
for the matrix-vector product \verb`A*x`. This ingenious strategy
of determining the output type only at assignment time is made possible
in a static language like C++, where users are expected to specify
the output type of a variable when declared.\footnote{Strictly speaking, C++ also allows for run-time polymorphism, where
the output type is only determined at run time. However, the mechanisms
for doing so are cumbersome and in most cases produce significant
performance slowdowns.} The result is that in almost all cases, users can choose to imagine
the result of most linear algebra expressions as vectors or matrices
as they so choose.

Nevertheless, Eigen's strategy is not perfect. The heavy use of internal
classes to represent the output of computations means that both assignment
and equality testing in Eigen must be overloaded with new definitions
to meet user expectations. Oftentimes, the validity of an assignment
statement can only be determined at run time, resulting in overhead
due to run time checks for compatible shape and so on. Many new methods
have to be introduced to overload C++'s assignment and equality testing
operators. Furthermore, the assignment logic of vector transpose is
ambiguous and may lead to subtle logic errors because the result of
assignment may violate user expectations of the transitivity of equality.
In Eigen, the result of \verb`x.transpose()` can be assigned to either
a \verb`VectorXd` variable or \verb`MatrixXd` variable, i.e., both
of the following statements are valid:
\begin{verbatim}

VectorXd y = x.transpose();
MatrixXd Y = x.transpose();

\end{verbatim}
However, the shapes of \verb`y` and \verb`Y` differ: the former
has $n$ rows and 1 column\footnote{Eigen defaults to column vectors and employs an implicit trailing
singleton dimension rule similar to \textsc{{MATLAB}'s.}} whereas the latter has 1 row and $n$ columns. In other words, the
type of the variable being assigned to determines whether transposing
a vector is the identity operation or produces a row matrix!

We find that unexpected behavior may result from the logic above.
For example, the identity ${(Ax)}^{\prime}=x^{\prime}A^{\prime}$
may or may not hold depending on whether the expressions on either
side have been assigned to variables, and if so, which variables:
\begin{verbatim}

//The type of (A*x).transpose() is
//  Transpose<GeneralProduct<MatrixXd, VectorXd>>
VectorXd v1 = (A*x).transpose();
MatrixXd m1 = (A*x).transpose();

//The type of x.transpose() * A.transpose() is
//  ProductReturnType<Transpose<VectorXd>, Transpose<MatrixXd>>
VectorXd v2 = x.transpose() * A.transpose();
MatrixXd m2 = x.transpose() * A.transpose();

assert ((A*x).transpose() == x.transpose() * A.transpose()); //true
assert (v1 == v2); // true
assert (m1 == m2); // true
assert (v1 == m2); // assertion failure
assert (m1 == v2); // assertion failure

\end{verbatim}
The equality comparison operator \verb`==` must be suitably defined
to understand when two values with complicated and different types
can test equal to each other in the three cases above that do not
trigger the assertion failure. Furthermore, the result of assigning
\verb`(A*x).transpose()` to \verb`v1` produces a column vector effectively
equal to \verb`A*x`, but the assignment to \verb`m1` produces a
row matrix that is the transpose of \verb`v1`. A similar result arises
for \verb`v2` and \verb`m2`, with the result that the equality holds
only when neither side is assigned to a vector or matrix variable,
or when both sides are assigned to variables of the same type. Note
that in Eigen, a row matrix is not assignable to a vector variable.
In this case, users may be confused as to why certain quantities may
not be equal to each other as a result of subtle reasoning errors
about the types of various expressions.

Eigen's semantics may be summarized as follows:
\begin{itemize}
\item \verb`.transpose()`: $F^{m\times n}\rightarrow F^{n\times m}$ is
$\cdot^{T}$.
\item \verb`.transpose()`: $F^{m}\rightarrow F^{m}$ is the identity map
$I_{v}$.
\item \verb`.transpose()`: $F^{m}\rightarrow F^{1\times m}$ is $\tilde{\phi}=\cdot^{T}\circ\phi$.
\item \verb`*`: $F^{m\times k}\times F^{k\times n}\rightarrow F^{m\times n}$
is matrix multiplication $*_{mm}$.
\item \verb`*`: $F^{m\times n}\times F^{n}\rightarrow F^{m\times1}$ is
the composition $*_{mm}\circ\left(I\times\phi\right)$.
\item \verb`*`: $F^{m\times n}\times F\rightarrow F^{m\times n}$ is $\circledast_{ms}$.
\item \verb`*`: $F\times F^{m\times n}\rightarrow F^{m\times n}$ is $\circledast_{sm}$.
\item \verb`*`: $F^{m}\times F^{1\times n}\rightarrow F^{m\times n}$ is
the outer product, $*_{mm}\circ\left(\phi\times I\right)$.
\end{itemize}
For every allowed method of \verb`*` producing $F^{m\times1}$ there
is an analogous method yielding $F^{m}$ which adds a final composition
$\phi^{-1}\circ$ to the computation. There are no corresponding method
of \verb`*` producing $F^{m}$ when the result produces row matrices
$F^{1\times m}$. In other words, $\tilde{\phi}^{-1}$ does not exist
in Eigen. However, $\tilde{\phi}$ DOES exist, but only as part of
the semantics of transpose.

\subsection{Julia}

Julia is a dynamic for technical computing, whose base library features
generous functionality for linear algebraic operations.~\cite{Bezanson2012}

Julia's multiplication operator, \verb`*`, implements matrix–matrix
and matrix–vector products, as well as scalings of vectors and matrices,
on top of ordinary scalar multiplication. Julia's division \verb`x/y`
is equivalent to multiplication of \verb`x` by the inverse of \verb`y`
on the right. For non-scalar \verb`y`, the inverse is generalized
to matrix inverses and pseudoinverses (in the least-squares sense)
where appropriate. Julia's transposition operator, \verb`'`, promotes
vectors to column matrices before transposing them. Thus we have that
vector transposition is not idempotent, \verb`x''`$\ne$\verb`x`.

Julia's current semantics lead to some unexpected behavior. Expressions
like \verb`a'*b` for the inner product produce a vector of length
1, which do not have the semantics of scalars. Other identities, like
$(xy^{\prime})z=x(y^{\prime}z)$, also fail to hold in code, since
\verb`x*(y'z)` becomes the product of a vector with a 1-vector, which
is not defined.

Julia's semantics are:
\begin{itemize}
\item Scalars $F$, vectors $F^{m}$ and matrices $F^{m\times n}$.
\item .': $F^{m\times n}\rightarrow F^{n\times m}$ is matrix transpose
$\cdot^{T}$.
\item .': $F^{m}\rightarrow F^{1\times n}$ is the composition $\cdot^{T}\circ\phi$.
\item ': $F^{m\times n}\rightarrow\left(F^{*}\right)^{n\times m}$ is matrix
transpose with complex conjugation $\cdot^{\prime}=\cdot^{*}\circ\cdot^{T}$,
and similarly for vectors and scalars
\item \verb`*`: $F^{m\times k}\times F^{k\times n}\rightarrow F^{m\times n}$
is matrix multiplication $*_{mm}$.
\item \verb`*`: $F^{m\times n}\times F^{n}\rightarrow F^{m}$ is the composition
$\phi^{-1}\circ*_{mm}\circ\left(I\times\phi\right)$.
\item \verb`*`: $F^{m}\times F^{m\times n}\rightarrow F^{m\times n}$ is
an error except when $m=1$, when it is the composition $*_{mm}\circ\left(\phi\times I\right)$
producing a $1\times1$ matrix.
\end{itemize}
As a result,
\begin{itemize}
\item u'{*}v$=\left(*_{mv}\circ\left(\cdot'\times I_{v}\right)\right)\left(u\times v\right)$
returns the inner product \uline{stored as a $1$-vector}
\item u{*}v'$=\left(*_{mv}\circ\left(I_{v}\times\cdot'\right)\right)\left(u\times v\right)$
returns the outer product as expected.
\item u'{*}A{*}v=(u'{*}A){*}v$=\left(*_{mv}\circ\left(*\times I_{v}\right)\circ\left(\cdot'\times I\times I_{v}\right)\right)\left(u\times A\times v\right)$
returns the bilinear form \uline{stored as a $1$-vector}
\item u''$=\left(\cdot'\circ\cdot'\right)\left(u\right)=\left(\cdot^{T}\circ\cdot^{T}\circ\phi\right)\left(u\right)=\phi\left(u\right)$
is not idempotent.
\end{itemize}
Not symmetic in treatment of $\phi^{-1}$ and $\phi$.

\subsection{Torch/Lua}

Torch is a Lua\cite{Lua} package for machine learning applications.~\cite{Torch}
Although not designed for linear algebra per se, Torch uses numerical
linear algebra routines heavily under the hood for fast computations.

The key data structure of Torch is its \verb`Tensor` class, which
implements the semantics of multidimensional arrays. Unlike the other
languages considered here, Torch only defines transposition on matrices
(tensors with two dimensions). As a result, most of the analysis about
expressions and identities using vector transposes cannot be applied,
as any expression involving a vector transpose produces an error.
Instead, Torch requires the \verb`dot` method to write expressions
such as the inner product, \verb`x:dot(y)`.

Semantics:

\verb`x:dot(y)` is now the composition $\theta\circ*_{mm}\circ\left(\tilde{\phi}\times\phi\right)$
that computes the (nonconjugated) inner product

A:t() : $F^{m\times n}\rightarrow F^{n\times m}$ is the matrix transpose,
$\cdot^{T}$. :t() is syntax sugar for the general :transpose(1, 2)
method, which is a permutation swapping the dimensions 1 and 2.

A:transpose(1) : $F^{n}\rightarrow F^{n}$ is the identity, $I_{v}$.

Another quirk of Torch is that binary expressions of a scalar and
a tensor must have the tensor operand come first, due to limitations
of the underlying Lua language, which does not allow packages to define
new methods on built-in types such as floating-point numbers. Thus
care must be taken to write scalar–matrix products in matrix–scalar
order.

\subsection{R}

R is a widely used language for statistical computing.~\cite{Rlang}.
R's statistical features work primarily on data frames, time series
and vectors. R also has matrices, but no true scalars. Its matrix
multiplication operator is \verb`%*%` and its transposition operator
is \verb`t()`. R's semantics promotes $n$-vectors automatically
to matrices whenever linear algebraic operations like \verb`t()`
or \verb`%*%` are encountered. However as we see later, $n$-vectors
behave either as $n\times1$ column matrices or $1\times n$ row matrices
depending on the context.

Extraneous singleton dimensions can be removed with \verb`drop()`,
which converts either $n\times1$ or $1\times n$ matrices into $n$-vectors.

R's semantics are:
\begin{itemize}
\item Only vectors $F^{m}$ and matrices $F^{m\times n}$.
\item \verb`t()`: $F^{m\times n}\rightarrow F^{n\times m}$ is matrix transpose
$\cdot^{T}$.
\item \verb`t()`: $F^{m}\rightarrow F^{1\times m}$ is the row matrix embedding
$\tilde{\phi}=\cdot^{T}\circ\phi$.
\item \verb`%*%`: $F^{m\times k}\times F^{k\times n}\rightarrow F^{m\times n}$
is matrix multiplication $\cdot^{T}$.
\item \verb`%*%`: $F^{m\times n}\times F^{n}\rightarrow F^{m\times1}$
is the matrix–vector product, $*_{mm}\circ\left(I\times\phi\right)$.
\item \verb`%*%`: $F^{m\times n}\times F^{1}\rightarrow F^{m\times n}$
is the matrix–scalar product, $\circledast_{ms}\circ\left(I\times\psi\right)$,
where $\psi=\theta\circ\phi$ is the vector-to-scalar embedding.
\item \verb`%*%`: $F^{m}\times F^{m\times n}\rightarrow F^{1\times n}$
is the vector–matrix product, $*_{vm}=*_{mm}\circ\left(\tilde{\phi}\times I\right)$.
Note that the first argument is promoted to a row matrix.
\item \verb`%*%`: $F^{m}\times F^{1\times n}\rightarrow F^{m\times n}$
is the outer product, $*_{mm}\circ\left(\phi\times I\right)$. Note
that the first argument is promoted to a column matrix.
\item \verb`%*%`: $F^{m}\times F^{m}\rightarrow F^{1\times1}$ is the vector–vector
product, $*_{mm}\circ\left(\tilde{\phi}\times\phi\right)$.
\item \verb`%*%`: $F^{1}\times F^{m\times n}\rightarrow F^{m\times n}$
is the scalar–matrix product, $\circledast_{sm}\circ\left(\psi\times I\right)$.
\item \verb`%*%`: $F^{1}\times F^{m}\rightarrow F^{1\times m}$ is the
scalar–vector product, $\circledast_{sm}\circ\left(\psi\times\tilde{\phi}\right)$.
Note that the analogous operation over $F^{m}\times F^{1}$ is not
defined.
\item \verb`drop()`: $F^{n\times1}\rightarrow F^{n}$ is the $\phi^{-1}$
embedding.
\item \verb`drop()`: $F^{1\times n}\rightarrow F^{n}$ is the $\tilde{\phi}^{-1}$
embedding.
\end{itemize}
As you can see, the \verb`%*%` operator in R is the most overloaded
operator in the set we are considering. On some level, \verb`%*%`
is simple in that its output is always a matrix. However, the embeddings
implicitly applied to vector arguments are by far the most complicated
of all the languags we have studied. Since R lacks true scalars, products
involving over vectors of length 1 implicitly $\psi$-embed them into
scalars. Depending on the context, vectors of other lengths undergo
either a $\tilde{\phi}$- or $\phi$-embedding, depending on context.

The results are
\begin{itemize}
\item u''$=\left(\cdot'\circ\cdot'\right)\left(u\right)=\left(\cdot^{T}\circ\cdot^{T}\circ\phi\right)\left(u\right)=\phi\left(u\right)$
is not idempotent. Nevertheless, we've already seen precedent in the
mathematical literature that does so... .
\end{itemize}
