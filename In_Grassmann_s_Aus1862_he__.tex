In Grassmann's Aus1862 he describes general structures to form products of two or more vectors. In modern language, Chapter 2 begins by saying that if you express two vectors in some basis

\begin{aligned}
u = \sum_i c_i e_i \\
v = \sum_i d_i e_i \\
\end{aligned}

and you have some vector product operation $\circ$ you want to define, then it suffices to describe the result of the operation $\circ$ on the basis vectors $e_i$, since 

\[
u \circ v = \sum_{ij} c_i d_i (e_i \circ e_j)
\]

If $e_i \circ e_j = \delta_{ij}$, then one recovers the ordinary inner product (Grassmann calls it that), i.e. $\circ = \cdot$.

If $e_i \circ e_j = - e_j \circ e_i$, then one recovers Grassmann's exterior product (which Grassmann calls the combinatorial product). Today, we would write $\circ = \wedge$ for this product.

Grassmann does not give the general product $\circ$ a special name, instead referring to it informally in the text defining his notation:

,,Ein Produkt, in welchern die Faktoren $a, b, \cdots$ irgend wie enthalten find, werde ich[...] mit $P_{a,b,...}$ bezeichnen`` \cite[p. 24, \S 43]{Grassmann1862}

``A product in which the factors a, b, ... are included in any way I will[...] denote by $P_{a,b,...}$''~\cite[p. 22, \S 43]{Grassmann2000}

Later in his discussion of symmetric tensors, Grassmann refers to ``ein beliebiges Produkt'' $P_{a_1, a_2, ..., a_n}$  (``an arbitrary product''~\cite[p. 196, \S 353]{Grassmann2000}), which Gibbs in his Multiple Algebra (collected works, Vol 2, p 109) picked up on as being of significance and bequeathed the name ``indeterminate product'', being the most general product from which all the other products Grassmann discusses can be derived.

Grassmann's other contribution to this topic is the concept of ,,offne Produkt`` or ``open product''~\footnote{Somewhat confusingly, Grassmann's 1862 book defines open products, or ``product[s] with n {interchangable openings''~\cite[\S\S 353, p. 196]{Grassmann2000}, in a way which we would recognize today as symmetric tensors of rank $n$.}, which he writes in Ausdehnungslehre 1844, Sec 172, p 267 (English pp 271-2) with the notation $[A() . B]$, acting on a vector $P$ by

\[
[A() . B] P = AP . B,
\]

or in modern notation,

\[
(b a^T) p = (a\cdotp) b,
\]

transcribing vectors into lower case letters in line with Householder's convention. In other words, Grassmann's open product is what we would call today the outer product, where ``open'' refers to the presence of the empty parenthesis denoting an ``opening''. Gibbs recognized the open product as a matrix in Multiple Algebra, Collected Works vol 2, p 94, but it is specifically a rank 1 matrix. (\textbf{TODO} Possibly the notion of rank did not exist then...?) 

In Gibbs's lecture notes of 1884, Sec. 107, p. 53 of Collected Works v2, he names the result of the indeterminate product of two vectors (in three dimensions) a dyad, and the linear combination of three dyads a dyadic, or what we would call today a 3x3 matrix. c.f. published version in Wilson1901.