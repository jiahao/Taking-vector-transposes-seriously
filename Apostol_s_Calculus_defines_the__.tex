\cite[p. 91]{Apostol1969} defines the matrix transpose elementwise, which is perhaps the more familiar form

The transpose of an $m \times n$ matrix $A = \left( a_{ij} \right)^{m,n}_{i,j=1}$ is the $n \times m$ matrix $A^t$ whose $i, j$ entry is $a_{ji}$.

The adjoint shows up on \cite[p. 122]{Apostol1969}

Apostol uses ``column matrix'' and ``column vector'' synonymously: \cite[p. 592]{Apostol1967} conflates the notions of tuple, array, vector and matrix (!):

We shall display the $m$-tuple $(t_{1k}, \dots, t_{mk})$ vertically... [t]his array is called a \textit{column vector} or a \textit{column matrix}.

If we interchange the rows and columns of a rectangular matrix $A$, the new matrix so obtained is called the \textit{transpose} of $A$ and is denoted by $A^t$. \cite[p. 615, Exercise 7]{Apostol1967}.

(Another axiom: $(A^t)^{-1} = (A^{-1})^t$)