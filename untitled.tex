\section{Definitions of the transpose}

Let's have a look at how the matrix transpose is defined in the mathematical literature.

[Gantmacher, vol. 1, p. 280, Definition 10] defines the transpose to have the following properties;

For a linear operator $A$ over $\mathbb R^n$,

the transpose $A^T$ is the operator such that for any two vectors $x, y \in \mathbb R^n$,

\[
\left\langle Ax, y \right\rangle = \left\langle x, A^T y \right\rangle
\]

and has the properties

\begin{align}

(A^T)^T & = A \\
(A + B)^T & = A^T + B^T \\
(\alpha A)^T & = \alpha A^T \text{ for all real numbers } \alpha \\
(A B)^T & = B^T A^T

\end{align}

Analogous definitions and properties for the $A^*$, the adjoint of $A$ (for operators over unitary space $\mathbb C^n$), appear on p. 266.