\subsection{The outer product}

The outer product is a misfit in the original list given in the introduction --- it is the only quantity which users expect to be a matrix, rather than a scalar.

First we need to address the fact that the outer product has two meanings in linear algebra. One sense is synonymous with the exterior product or wedge product, and was the original ``\textit{äußeres Produkt}'' of Grassmann's \textit{Ausdehnungslehre}.~\cite{Grassmann1862,Grassmann1877,Grassmann1995,Grassmann2000} The other sense is the tensor product of two vectors, which first appeared in Einstein and Grossmann's 1913 paper introducing general relativity,~\cite{Einstein1913} where they write

Die de gewöhnlichen Vektoranalsysis entnommenen Bezeichnungen ,,äußeres und inneres Produkt`` rechtfertigen sich, weil jene Operationen sich letzten Endes als besondere Fälle der hier betrachteten ergeben. The designations "inner and outer product", which are taken from ordinary vector analysis, are justified because, when all is said and done, those operations prove to be special cases of the operations considered here.~\cite[T. II, p. 26]{Einstein1913,Einstein1996}



The term ``\textit{äußeres Produkt}'' continued to be used widely in physics. The second edition of Hermann Weyl's book ``\textit{Gruppentheorie und Quantenmechanik}''~\cite{Weyl1931,Weyl1950} contained the phrase, \textbf{TODO Confirm this fact!} even though it was absent in the first edition.

\textbf{TODO Did Jordan, 1927 in the quantum mechanics papers do this too?}

The term ``outer product'' in this sense was introduced into the English literature in books about general relativity, such as in...

The term ``outer product'' appears to have been rediscovered by Ken Iverson in his seminal publications about APL. In ``\textit{A Programming Language}''~\cite{Iverson1962book}, 

We should note that the ``outer product'' in the sense of tensor product has existed in many different forms throughout the mathematical literature, and has been called many different names and has many different notations. The oldest mention we could find of this concept is Grassmann's indeterminate product.

...

Grossmann cited Ricci and Levi-Civita's paper on tensor analysis,~\cite{Ricci1900} where the multiplication of two ``systèmes covariants (covariant systems)'' produced what was simply called a ``produit (product)''~\cite[p. 133]{Ricci1900} and is annotated as ``(tensor) product'' in a translation by Hermann.~\cite[p. 28]{Hermann1975}.

Apparently independently, Gibbs's dyads...


The first modern expression of the outer product $u v^T$ as the product of a vector and another transposed vector appears to be in Bodewig 1956\cite{Bodewig1956}, wherein it was called the simple product. \textbf{TODO Check first edition; it's in the second!}
