\label{sec:array_history}
\section{Vectors in computer science}

In computer science, vectors are synonymous with one-dimensional arrays, or simply arrays.~\cite{Knuth1967,Pratt2001} The array data structure is
defined algebraic objects, vectors are elements of a vector space and matrices are linear transformations.

\subsection{A brief history}

To computer scientists, a vector is a data structure that models of random access memory. For example, SICP describes

\begin{quote}
A conventional computer memory can be thought of as an array of cubbyholes, each of which can contain a piece of information. Each cubbyhole has a unique name, called its \textit{address} or \textit{location}. Typical memory systems provide two primitive operations: one that fetches the data stored in a specified location and one that assigns new data to a specified location. Memory addresses can be incremented to support sequential access to some set of the cubbyholes. More generally, many important data operations require that memory addresses be treated as data, which can be stored in memory locations and manipulated in machine registers. The representation of list structure is one application of such \textit{address arithmetic}.

To model computer memory, we use a new kind of data structure called a \textit{vector}. Abstractly, a vector is a compound data object whose individual elements can be accessed by means of an integer index in an amount of time that is independent of the index. In order to describe memory operations, we use two primitive Scheme procedures for manipulating vectors:

\begin{itemize}
\item
\code{(vector-ref <vector> <n>)} returns the $n$th element of the vector.

\item
\code{(vector-set! <vector> <n> <value>)} sets the $n$th element of the vector to the designated value.
\end{itemize}--\cite[\S 5.3.1]{sicp}
\end{quote}

\subsection{More notes}

The seminal work of~\cite{Iliffe1961} says
%
\begin{quote}
``Depending on the organization of the array it may
be treated as a \textit{set}, a \textit{vector}, or a \textit{matrix}.''
\end{quote}


Computer science focuses primarily on the homogeneous container semantics of
vectors and matrices. Oftentimes they are considered synonymous with arrays of
rank 1 and 2 respectively.
%
The classic~\cite{Knuth1967} focuses only on indexing semantics; the index
entry for ``two-dimensional array'' cross-references the entry for ``matrix''.
Even today, the conflation persists. A modern textbook on programming language
design writes~\cite[p. 215]{Pratt2001}:
%
\begin{quote}
A vector is a one-dimensional array; a matrix composed of rows and columns of
components is a two-dimensional array[.]
\end{quote}

%PZ p 217 also has a nice description of linear indexing semantics adopted from
%Fortran. Slicing came from PL/I.


\paragraph{Brooker (1955)}

Describes ``indices'':

Indices. These correspond to the subscripts of conventional
mathematical notation. In order to take advantage
of the repetitive nature of calculation it is desirable to have
some means of specifying any of the elements of a sequence,
e.g. the components of a vector. For this purpose the notation
vnl, wn2, vn3, etc., is introduced where nl, n2, n3, etc., are
indices restricted to integral values but otherwise computable
like variables. 

@article{Brooker1955,
	Abstract = {This paper outlines some of the difficulties encountered in attempting to simplify the task of preparing programmes for a large scale digital computer. In general this is most simply achieved at the price of increasing the machine time consumed by the resulting programme. Such a compromise is nevertheless worth while for many "small" problems and a means is described for treating such problems on the Manchester University Computer (Ferranti MK I computer). The method is also applicable to any of the Ferranti range of machines, several of which are now available in this country.},
	Author = {R A Brooker},
	Journal = {British Journal of Applied Physics},
	Number = {9},
	Pages = {307},
	Title = {An attempt to simplify coding for the Manchester electronic computer},
	Url = {http://stacks.iop.org/0508-3443/6/i=9/a=302},
	Volume = {6},
	Year = {1955},
	Bdsk-Url-1 = {http://stacks.iop.org/0508-3443/6/i=9/a=302}}

\paragraph{Chipps et al. (1956), ``A Mathematical Language Compiler''}

Describes an early programming language for mathematical formulas.

They define scalars, vectors and matrices by the number of indexes each quantity
is allowed to take. (Cutely, they use \lstinline|x| for multiplication and
\lstinline|*| for exponentiation)

@inproceedings{Chipps:1956:MLC:800258.808963,
 author = {Chipps, J. and Koschmann, M. and Orgel, S. and Perlis, A. and Smith, J.},
 title = {A Mathematical Language Compiler},
 booktitle = {Proceedings of the 1956 11th ACM National Meeting},
 series = {ACM '56},
 year = {1956},
 pages = {114--117},
 numpages = {4},
 url = {http://doi.acm.org/10.1145/800258.808963},
 doi = {10.1145/800258.808963},
 acmid = {808963},
 publisher = {ACM},
 address = {New York, NY, USA},
}



\paragraph{Hockney (1961)~\cite{Hockney1961}}

A 1961 paper by Hockney in the ALGOL Bulletin
describes the modern concept of the array data structure as describing
both elements and shape. Para. 3.7:

``This concept of <array type> requires that an array carries with it a
statement of its dimensions and the dimensions of any of its elements
that may themselves be arrays... One should think of an array as the
totality of the elements of the array together with all the bound pair
lists describing the dimensions of the array.''

The term ``array'' appears to have evolved from its vernacular usage in ``array of cobscripted variables''

Has the first description of a multidimensional array as parametric on indexing quantities and the element type.


\paragraph{Strachey (1967)~\cite{Strachey1967}}

By the time computer scientists thought of formalizing the grammar and semantics of programming languages,
vectors and arrays are considered well-understood data structures. \cite[\S 3.7.7, pp. 43--44]{Strachey1967} writes:
\begin{quote}
Vectors and arrays are reasonably well understood. They are parametric types so that the type of an array includes its dimensionality (the number of
its dimensions but not their size) and also the type of its elements[...]
all the elements of an array have to be of the same type, though their number may vary dynamically.
It is convenient, though perhaps not really necessary, to regard an $n$-array
(i.e., one with $n$ dimensions) as a vector whose elements are ($n-1$)-arrays.
We can then regard the R-value of a vector as something [that] gives access (or points to) the elements rather than containing them.
\end{quote}
We recognize herein the modern description of the array data type being defined by
two parameters: the number of dimensions and the type of elements.

Does Strachey assume integer indexes?


\paragraph{Wirth (1973)~\cite{Wirth1973}}

The book ``Structured Programming'' is possibly the first major textbook on theoretical computer science, as it introduces the study of data structures and algorithms as a discipline separate from mathematics.

Wirth uses a pseudocode notation inspired by ALGOL 60. He also credits Rutishauser
in the preface as ``the originator of the idea of programming languages''.

This book is notable for Ch. 11, about the array data structure, and Ch. 8, on data types and types.

Types: Wirth introduces p44 that ``The introduction of a new variable, in particular,
should be accompanied by the specification of
its range of possible values.'' Wirth states that doing so is necessary to understand
how an algorithm works and its correctness, and further allows reasoning about memory requirements
of efficient representations.
Wirth defines ``type'' informally as ``[t]he set of values that a variable may
assume''. p45 also introduces the concept of ``scalar value'' as a value
which is ``unstructured (i.e.,\ not decomposable into components.)''


Wirth defines the array data structure as `` a collection of \textit{component
variables of the same type}'' - Ch. 11, p. 80, being one of his primary
examples of a structured type. As distinct from a file, ``Each single component
of an array is explicitly denotable and directly accessible''.\footnote{Wirth
also says that the number of
components of an array is fixed once initialized, which is something we don't
consider to be an essential characteristic of arrays anymore in the context of
mutable lists.}

Wirth says that an array type must specift rhe componen type and the index type,
generalizing Strachey's definition to introduce possibly non-integer indexes.

p 83 - gives as an example a program to compute the scalar product.
Wirth does not refer to the arrays as vectors, but simply lists the elements of
both vectors, each with a single subscript.

p 87 - ``Components of arrays need not be scalars---they themselves may be structured.
If they are again arrays, then the orrignal array $A$ is called \textit{multi-dimensional}.
If the components of the componet arrays are scalars, then
$A$ is called a \textit{matrix}.'' Wirth clearly defines arrays recursively.


\paragraph{Array theory} The APL literature contains a comprehensive discussion
of multidimensional array indexing semantics~\cite{Brown1982}, which were
absent in the classical APL of \cite{Iverson1962}. \cite{Ruehr1982,Gerth1988}
surveyed the development of array semantics in APL.

What we refer to as APL style indexing was first written down in the APL/360
manual~\cite{Falkoff1968}, albeit only for rank 2 only, where indexing by a
matrix returned a matrix. \cite{Haegi1976} noted an inconsistency. The general
case has been termed ``scatter-point indexing'' or ``choose indexing''
\cite{Brown1972,Ruehr1982}, and was perhaps most clearly described
in~\cite{More1979}. \cite{Gull1979} emphasized the importance of preserving
the rank of the index set in the rank of the output for multidimensional
arrays.

\cite{More1973} built a theory of multidimensional arrays using recursive
nesting of one-dimensional arrays, formalized upon axiomatic set theory.
\cite{Ghandour1973} emphasized the ordering of elements as a defining
characteristic of arrays. \cite{Gerth1988} emphasized indexing as a function on
arrays from index sets to value sets. We do not consider functional aspects of
indexing in this paper.

The APL literature makes some mention of linear algebra semantics, but does not
consider the interplay with array indexing semantics.

\begin{enumerate}
\item \cite{More1973} mentions the inner product returning a scalar as an
	important rule for guiding array semantics.

\item \cite{Haegi1976} bemoans the conflation of scalars and $(1,)$-arrays in
	classical APL.

\item \cite[p. 153]{More1973} also suggests the possibility of more complicated
	multidimensional array semantics:

\begin{quote}
Let $V$ be an $n$-dimensional vector space over a field. Disregarding
considerations of contravariance and covariance, a tensor of valence $q$ on $V$
is a multilinear mapping of the Cartesian product of the list $V V \dots V$ of
length $q$ into a vector space. If $V$ has a base, then a tensor of valence $q$
on $V$ can be represented by a \textit{component tensor}, which is an array on
$q$ axes, each of length $n$.
\end{quote}

\end{enumerate}


\paragraph{Becker, Chambers and Wilks (1988)~\cite{Becker1988}}

The definitive book introducing the statistical programming language S in its
modern form, which was adapted by its open source implementation, R.

p. 125: Classes of objects in S

Class | Use
vector | simple objects
matrix | 2-dimensional array
array | multi-way array

Notable for saying on Ch. 5, p. 128

\begin{quote}
If the matrices resulting from subscripting end up with one row or one
column, there is a chocie to make. S can retain the matrix properties or it can
drop one dimension to rpoduce a vector. By default, S drops these redundant,
or ``dead,'' dimensions. [...]
In most cases of interactive computing, that is the right choice since it tends to
reuduce complexity---why have a 1-column matrix when a vector will do.
\end{quote}

which is not mentioned in the 1984 book or the 1985 book.
