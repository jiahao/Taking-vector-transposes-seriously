\subsection{The contributions of Grassmann}

The contribution of Grassmann to the user psychology of linear algebra are not well understood, even today. In this section, we survey the contributions of Grassmann to our terminology.

It's well known that Grassmann introduced the notions of inner and outer products; less known perhaps is the history of the term ``outer product''.

TODO put text about the two kinds of outer product here


Grassmann's \textit{Ausdehnungslehre} of 1862\cite[Ch. 2]{Grassmann1862} introduces describes general structures to form products of two or more vectors. In modern language, if two vectors are expressed in some basis

\begin{aligned}
u & = \sum_i c_i e_i \\
v & = \sum_i d_i e_i \\
\end{aligned}

then any vector product operation $\circ$ is fully defined if the result of the operation $\circ$ on the basis vectors $e_i$ is known, since 

\[
u \circ v = \sum_{ij} c_i d_i (e_i \circ e_j) = \sum_{ij} c_i d_i G_{ij}
\]

where we introduce formally $G_{ij}$, the Gramian of the vector space with basis ${e_i}$ and inner product $\circ$.

Grassmann defined the inner product to be the product whose Gramian is the identity, and the outer product by placing new basis vectors in each uniquely defined entry of an antisymmetric $G$. Interestingly, Grassmann does not give the general case a name, instead referring to it informally in the text defining his notation:

,,Ein Produkt, in welchern die Faktoren $a, b, \cdots$ irgend wie enthalten find, werde ich[...] mit $P_{a,b,...}$ bezeichnen`` \cite[p. 24, \S 43]{Grassmann1862}

``A product in which the factors a, b, ... are included in any way I will[...] denote by $P_{a,b,...}$''~\cite[p. 22, \S 43]{Grassmann2000}

Gibbs, however, recognized the value of the general case, bequeathing it the name ``indeterminate product''.~\footnote{Some of the English literature incorrectly attribute the name to Grassmann; however, it is quite clear from \textit{Multiple Algebra} that the name is his invention. The closest phrase used by Grassmann is ,,ein beliebiges Produkt $P_{a_1, a_2, ..., a_n}$`` (``an arbitrary product''~\cite[p. 196, \S 353]{Grassmann2000}), which is not defined in a formal, technical sense.} If we interpret this as using an unsymmetric Gramian $G$ and placing a new unique basis vector in each entry of $G$, then we have the basic ingredients of a tensor product.

Grassmann's other contribution to this topic is the concept of ,,offne Produkt`` or ``open product''~\footnote{Somewhat confusingly, Grassmann's 1862 book defines open products, or ``product[s] with n \{interchangable\} openings''~\cite[\S\S 353, p. 196]{Grassmann2000}, in a way which we would recognize today as symmetric tensors of rank $n$.}, which he writes in Ausdehnungslehre 1844, Sec 172, p 267 (English pp 271-2) with the notation $[A() . B]$, acting on a vector $P$ by

\[
[A() . B] P = AP . B,
\]

or in modern notation,

\[
(b a^T) p = (a\cdot p) b,
\]

transcribing vectors into lower case letters in line with Householder's convention. In other words, Grassmann's open product is what we would call today the outer product, where ``open'' refers to the presence of the empty parenthesis denoting an ``opening''. Gibbs recognized the open product as a matrix in Multiple Algebra, Collected Works vol 2, p 94, but it is specifically a rank 1 matrix. (\textbf{TODO} Possibly the notion of rank did not exist then...?) 

In Gibbs's lecture notes of 1884, Sec. 107, p. 53 of Collected Works v2, he names the result of the indeterminate product of two vectors (in three dimensions) a dyad, and the linear combination of three dyads a dyadic, or what we would call today a 3x3 matrix. c.f. published version in Wilson1901.