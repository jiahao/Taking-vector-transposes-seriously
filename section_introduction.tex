\section{Introduction}

The thesis of this paper is as follows:

Users want to write, using compositions of the transpose and product operators, linear algebra expressions that involve vectors and matrices, and the results make sense, i.e. the results of \texttt{u'*v} behaves like a scalar.Examples of linear algebraic expressions users expect to write include:

\begin{description}

\item{Inner product} $u^T v$, a scalar,
\item{Outer product} $u v^T$, a matrix,
\item{Quadratic form} $u^T A u$, a scalar,
\item{Bilinear form} $u^T A v$, a scalar,

\end{description}

The description of these quantities employ Householder notation~\cite{Householder1953,Householder1955}, a convention that is familiar to most, if not all, practitioners of numerical linear algebra today.



\subsection{Introduction (v1)}

Humans, to a fault, are so good at special casing abstractions
by context that it would be easy to conclude that a discussion
of how linear algebra fits into computer languages would hardly
seem necessary.  Decades after APL made multidimensional
arrays first class objects, and MATLAB made matrix laboratory
syntax popular,  we recently  reached an inescapable conclusion,
one we preferred not to believe, that
vectors, in the sense of computer languages,
and linear algebra vectors have serious coexistence issues.



To set the stage, in nearly every computer language (other than MATLAB (!!) )
a ``vector"  is a one dimensional container.  In concrete
linear algebra there are column vectors and row vectors.
In  both concrete and abstract linear algebra, vectors are quantities that are subject
to addition, multiplication by scalars, linear transformations, and
play a role in scalar and outer products.
As we shall see, the  seemingly innocuous idea of transposing a vector turns out to be
a highly contentious subject.


Consider the following familiar  statements  \begin{itemize}
\item A matrix times a vector is a vector
\item Matrix Multiplication of $A$ and $B$ is defined by taking the scalar product
of a row of $A$ with a column of $B$.
\item  The scalar product is defined on a pair of vectors.
The scalar product of  $v$ with itself is a non-negative number known as
$\|v\|^2$.
\end{itemize}

and consider common  notations in math or software:
\begin{itemize}
\item  \verb+A[i,:]*B[:,j]+ or \verb+dot(A[i,:],B[:,j]) +
\item   $v^T\!v=(v,v)=$ \verb+v'*v+=\verb+dot(v,v)+
\end{itemize}

(more coming)



\subsection{Introduction (v0)}

Matrices and vectors are fundamental concepts for both computer science
~\cite{Knuth1967,Pratt2001} and computational
science~\cite{Strang2003,Trefethen1997}. Nevertheless, the terms ``vector'' and
``matrix'' mean different things in the contexts of data structures and linear
algebra. As data structures, they are simply arrays, whereas as linear
algebraic objects, vectors are elements of a vector space and matrices are linear transformations.
When we represent linear algebra vectors and linear algebra transformation in a basis,
we obtain the familiar containers we know simply as vectors and matrices.



Computer science focuses primarily on the homogeneous container semantics of
vectors and matrices. Oftentimes they are considered synonymous with arrays of
rank 1 and 2 respectively. The seminal work of \cite{Iliffe1961} says
%
\begin{quote}
``Depending on the organization of the array it may
be treated as a \textit{set}, a \textit{vector}, or a \textit{matrix}.''
\end{quote}
%
The classic \cite{Knuth1967} focuses only on indexing semantics; the index
entry for ``two-dimensional array'' cross-references the entry for ``matrix''.
Even today, the conflation persists. A modern textbook on programming language
design writes~\cite[p. 215]{Pratt2001}:
%
\begin{quote}
A vector is a one-dimensional array; a matrix composed of rows and columns of
components is a two-dimensional array[.]
\end{quote}

%PZ p 217 also has a nice description of linear indexing semantics adopted from
%Fortran. Slicing came from PL/I.

In contrast, vectors and matrices in linear algebra are defined by their
algebraic properties, such as inner products and matrix-vector products.

The aim of this paper is to identify if and when the semantics of array
indexing and linear algebra may conflict.
